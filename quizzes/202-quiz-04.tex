\documentclass[12pt, oneside]{article}
\usepackage{soto-quiz}
\usepackage{tikz}
\usepackage{fancyhdr}

\newif\ifsolution
\input{solution}

\begin{document}

\lhead{ENSP 202}
\chead{Exercises 4}
\rhead{Monday 24 Mar 2014}
\lfoot{}
\cfoot{}
\rfoot{}
\pagestyle{fancy}

\namebox

Do this work in your class notes and write your answers on this sheet
and turn in at the end of class.

\problem{Ladders}

You need to reach a spot on a wall 9 meters above the ground.  You have
a 10 meter ladder.  If the top of the ladder reaches 9 meters up, how
far from the wall is the bottom?  First take an educated guess.  Then
use trigonometry.  Draw a picture and label the
distances.  Does this feel safe?

\solution{
There are multiple ways to solve this.

The sine of the angle the ladder makes with the wall is given by 9 over
10 or 0.9.

The angle that makes this sine can be calculated using the arcsin
function.  We get an angle about 65 degrees.  The distance from the
wall is given by the cosine of this angle multiplied by the length of
the ladder.  This is about 4.4 meters.

\begin{tikzpicture}

\coordinate (A) at (4.4,9);

\draw (0, 0) -- (A)
    node[pos=0, anchor=south west] {$\alpha$}
    node[pos=0.50, anchor=south, sloped] {10 meters};

\draw (0, 0) -| (A)
    node[pos=0.25, anchor=north] {4.4 meters}
    node[pos=0.75, anchor=west] {9 meters} ;

\end{tikzpicture}
}

\problem{Credit Card}

A credit card has a monthly interest rate of 2 percent.
The function that credit cards use to calculate your balance is

$$ \textrm{New Balance} = \textrm{Original Balance}
\cdot (1 + \textrm{interest rate})^{\textrm{Number of Periods}} $$

If you buy a
1000 dollar item and don't make a payment, what is your account balance
in one year?

2 years?

3 years?


\solution{
$$ 1000 \cdot (1 + 0.02)^{12} = 1286.42 $$
$$ 1000 \cdot (1 + 0.02)^{24} = 1608.43 $$
$$ 1000 \cdot (1 + 0.02)^{36} = 2039.89 $$
}

\problem{Exponential Growth}

The mass of mold on my pizza is 2 grams right now.  Mold scientists tell
me it will grow according to the function

$$ \textrm{Mold (grams)} = 2 e^{0.08 t} $$

Where $t$ is the number of hours from now.

How many grams of mold will
there be in a twelve hours?

\solution{

$$ 2 e^{0.08 \cdot 12} = 5.2 $$

5.2 grams or almost triple.

}

How long until I have 20 grams of mold?

\solution{
This requires a little bit of algebra.  The question we ask is, what is
$t$ when the amount of mold equals 20 grams?

$$ 20 grams = 2 e^{0.08 t} $$
$$ 10 = e^{0.08 t} $$

Recall what the natural log function tells us: If $y = e^x$ then $\ln y = x$

$$ ln 10 = 2.3 = 0.08 \cdot t $$
$$ t = 2.3/0.08 = 29 hours $$

}

\end{document}


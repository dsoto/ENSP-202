\documentclass{article}
\usepackage{soto-homework}

\newif\ifsolution
\input{solution}

\begin{document}

\chead{ENSP 202 Homework 3}
\chead{Due Date:  Mon 17 Feb 2014}
\hrule
\vspace{10pt}

For all the following examples, explain your strategy in a few
sentences, and clearly show your work including intermediate steps.
Solutions that only provide the answer without any supporting
explanation or calculations will not receive credit.  It should be clear
from your work that you understand the material.  Write your assignment
separate paper from this handout.  Please turn in your homework stapled
together.

{\tiny Last modified: \today}


\problem{Unit Conversion}
Show how to convert each of the following given units to the desired
units.

\subproblem
Convert 30 miles per hour to meters per second.

\solution{

$$ \ufrac{30 miles}{hour} \cdot
   \ufrac{1609 meters}{1 mile} \cdot
   \ufrac{1 hour}{60 minutes} \cdot
   \ufrac{1 min}{60 seconds}
   = 13.4 \ufrac{meters}{second}
$$
}

\subproblem
Convert 1 year to seconds.

\solution{
$$ \ufrac{1 year}{} \cdot
   \ufrac{365 days}{1 year} \cdot
   \ufrac{24 hours}{1 day} \cdot
   \ufrac{60 minutes}{1 hour} \cdot
   \ufrac{60 seconds}{1 min}
   = 3.15 \cdot 10^7 seconds
$$
}

\problem{Estimations}

These problems a practice in estimations using linear quantities.  We
will begin problems involving areas (squared quantities) and volumes
(cubed quantities) soon.  Clearly show each of the necessary quantities
for your calculation, their values, and a bit about how you made each
estimation.

\subproblem{Guesstimation Problem 3.2}

\solution{Available in text

The earth's circumference is about 40,000 km or $4 \cdot 10^7$ meters.
A golf ball is about 4cm in diameter or $4 \cdot 10^{-2}$ meters.

Dividing these two lengths gives the number of balls, which is $10^9$ or
about one billion.

}

\subproblem{Guesstimation Problem 3.6}

\solution{Available in text

I assume that a cat is about 5 kg in mass.  (I have a confession to
make.  I haven't weighed a cat
recently.)

We multiply Avagadro's number ($6 \cdot 10^{23}$) by 5 kg and get $3
\cdot 10^{24} kg$ which is about half the mass of the earth.

Note that this is also implies a large number of litter boxes.
}

\subproblem{Guesstimation Problem 3.7}

\solution{Available in text

I assume that the area of the card is about 10cm by 10cm or 100 square
centimeters.
For my estimation of the thickness, I use a pack of playing cards as a
reference.  A deck of cards is about 2 cm thick and there are about 50
cards.  1cm divided by 50 is 0.02 mm for each card.  We'll call this
half a mm.  Each cards volume is

$$ 10 cm \cdot 10 cm \cdot 0.002 cm = 0.2 cubic centimeters $$

If a card is the density of water, the card weighs 0.2 grams.  $10^8$ of
these tickets is $2 \cdot 10^7$ grams or $2 \cdot 10^3$ kg.  This is
about 200 tons.  5 40 ton trucks will be needed to carry 200 tons.

}

\subproblem{Guesstimation Problem 3.11}


\solution{Available in text

We assume that a student can put away 200 books per hour and works 40
hours a week for 3 weeks.

$$ \ufrac{200 books}{hour $\cdot$ student} \cdot
\ufrac{40 hours}{week} \cdot \ufrac{3 weeks}{} = \frac{2 \cdot 10^4
books}{student}$$

Since we need to shelve two million ($2 \cdot 10^6 books$) this is 100
students.

}

\subproblem{Metta World Pizza}

This problem follows our example from class.  When we have the pizza
delivered for our world pizza party, how high will the stack of pizza
boxes reach?

\solution{
Start by listing the quantities we need and the estimates

World population 7 billion\\
Slices per person 2\\
Slices per pizza 8\\
Box height per pizza 4 cm\\

We then check to be sure our units make sense.

$$ box height = persons \cdot \ufrac{slices}{person}
                \cdot \ufrac{pizza}{slice}
                \cdot \ufrac{box height}{pizza} $$

$$ box height = 7 \cdot 10^9 persons \cdot \ufrac{2 slices}{1 person}
                \cdot \ufrac{1 pizza}{8 slices}
                \cdot \ufrac{4 cm}{1 pizza} $$

$$ box height = 7 \cdot 10^9 cm$$

I'd rather express this in kilometers since it is a large number of a
small unit.

$$ 7 \cdot 10^9 cm \frac{1 m}{10^2 cm} \frac{1 km}{10^3 m}
= \boxed{7 \cdot 10^4 km} $$

}

\problem{Review}

\subproblem{Convert $123.12_5$ to a base 10 number.}

\solution{
I start by determining the value of each place.  In this case we have

$$ 5^2, 5^1, 5^0, 5^{-1}, 5^{-2} $$

As whole numbers and fractions, these are

$$ 25, 5, 1, 1/5, 1/25 $$

As decimal numbers, these are

$$ 25, 5, 1, 0.2, 0.04 $$

We multiply by the number occupying each place by the value of that
location and sum.

$$ 1 \cdot 25 + 2 \cdot 5 + 3 \cdot 1 + 1 \cdot 0.2 + 2 \cdot 0.04 = 38.28$$

}

\subproblem{Express 7,560,000,000 in scientific notation}

\solution{
$7.56 \cdot 10^9$
}

\problem{Final Project}

List three topics you find interesting that you will explore
quantitatively for your final project.

\problem{Time Spent}

Please estimate the amount of time you spent on this homework.

\end{document}

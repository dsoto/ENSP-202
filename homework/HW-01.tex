\documentclass{article}
\usepackage{soto-homework}

% set up whether we are printing assignment or solution
\newif\ifsolution
\input{solution}

\newcommand{\ufrac}[2]{\frac{\textrm{#1}}{\textrm{#2}}}

\begin{document}

\chead{ENSP 202 Homework 1}
\chead{Due Date:  Mon 03 Feb 2014}
\hrule
\vspace{10pt}


For all the following examples, explain your strategy in a few
sentences, and clearly show your work including intermediate steps.
Solutions that only provide the answer without any supporting
explanation or calculations will not receive credit.  It should be
clear from your work that you understand the material.

\problem{All your base are belong to us}

Express the following binary numbers in decimal

\subproblem $10110_2$ \solution{\boxed{22_{10}}

I start by writing out the value assigned to each of the places.  From
left to right, they are

$$ 2^4, 2^3, 2^2, 2^1, 2^0 $$
$$ 16, 8, 4, 2, 1 $$

I then multiply each place by the value in that place

$$ 1 \cdot 16 + 0 \cdot 8 + 1 \cdot 4
 + 1 \cdot 2 + 0 \cdot 1 = \boxed{22_{10}} $$
}

\subproblem $1_2$ \solution{\boxed{1_{10}}

The number 1 is the same in any base.  1 in the ones place equals 1 and
done.
}

\subproblem $10.01_2$ \solution{\boxed{2.25_{10}}

Now we encounter fractional places.  We calmly proceed by writing the
values of the places, noticing that they are also assigned to negative powers
now.

$$ 2^1, 2^0, 2^{-1}, 2^{-2} $$
$$ 2, 1, \frac{1}{2}, \frac{1}{4} $$

Then we multiply and add just as we did before

$$ 1 \cdot 2 + 0 \cdot 1 + 0 \cdot \frac{1}{2} + 1 \cdot \frac{1}{4} $$
$$ 2 + \frac{1}{4} = \boxed{2.25_{10}} $$


}

\subproblem $1100_2$ \solution{ \boxed{12_{10}}

$$ 2^3, 2^2, 2^1, 2^0 $$
$$ 8,4,2,1 $$
$$ 1 \cdot 8 + 1 \cdot 4 + 0 \cdot 2 + 0 \cdot 1 = \boxed{12_{10}} $$
}

\subproblem $101.1_2$ \solution{\boxed{5.5_{10}}

$$ 2^2, 2^1, 2^0, 2^{-1} $$
$$ 4, 2, 1, \frac{1}{2} $$

Then we multiply and add just as we did before

$$ 1 \cdot 4 + 0 \cdot 2 + 1 \cdot 1 + 1 \cdot \frac{1}{2} $$
$$ 5 + \frac{1}{2} = \boxed{5.5_{10}} $$
}



Express the following hexadecimal numbers in decimal.  Be sure you
understand the meaning of A, B, C, D, E, and F.

\subproblem $20C_{16}$ \solution{ \boxed{524_{10}}

I start by writing out the places recalling that hexadecimal uses 16 as
its base.  From left to right, they are

$$ 16^2, 16^1, 16^0 $$
$$ 256, 16, 1 $$

I then multiply each place by the value in that place, recalling that
since arabic numbers only have 10 symbols, A=10, B=11, C=12, D=13, E=14,
and F=15.

$$ 2 \cdot 256 +0 \cdot 16 + C_{16} \cdot 1 $$
$$ 2 \cdot 256 +0 \cdot 16 + 12 \cdot 1 = \boxed{524_{10}} $$
}


\subproblem $AA10_{16}$ \solution{\boxed{43536_{10}}

$$ 16^3, 16^2, 16^1, 16^0 $$
$$ 4096, 256, 16, 1 $$
$$ A_{16} \cdot 4096 + A_{16} \cdot 256 + 1 \cdot 16 +0 \cdot 1  $$
$$ 10 \cdot 4096 + 10 \cdot 256 + 1 \cdot 16 +0 \cdot 1 =
\boxed{43536_{10}}  $$
}

\subproblem $1AB_{16}$ \solution{ \boxed{427_{10}}

$$ 16^2, 16^1, 16^0 $$
$$ 256, 16, 1 $$


$$ 1 \cdot 256 + A_{16} \cdot 16 + B_{16} \cdot 1  $$
$$ 1 \cdot 256 + 10 \cdot 16 + 11 \cdot 1 = \boxed{427_{10}}$$
}

Write the following decimal numbers in binary notation:

\subproblem $8_{10}$ \solution{\boxed{1000_2}

Converting from decimal to binary is a little less straightforward.  To
start, we ask how many digits do we need to represent this number?  Or,
what is the biggest place we need?  We can write out a few places and
ask ourselves if we need them.

$$ 2^5, 2^4, 2^3, 2^2, 2^1, 2^0 $$
$$ 32, 16, 8, 4, 2, 1 $$

What digit do we put in the 32 place?  Zero since 8 is less than 32.
How about the 16 place?  Zero as well.

How about the 8 place?  Our number is 8 so we put a 1 in the 8 place and
there is nothing left over so the rest of the places are also zero.

$$ \boxed{1000_2} $$
}

\subproblem $32_{10}$ \solution{\boxed{100000_2}

$$ \boxed{100000_2} $$
}

\subproblem $123_{10}$ \solution{\boxed{1111011_2}

This number is larger than either of the last two examples so we'll need
more digits.  We start with

$$ 2^7, 2^6, 2^5, 2^4, 2^3, 2^2, 2^1, 2^0 $$
$$ 128, 64, 32, 16, 8, 4, 2, 1 $$

Do we need a one or a zero in the 128 place?  Zero since 123 is less
than 128.  How about the 64 place?  We place a one there and now we have
$123 - 64 = 59$ left to represent.  59 is greater than 32 so we also put
a one in the 32 place leaving $59 - 32 = 27$ left to represent.  We
continue in this pattern and find

$$ \boxed{1111011_2} $$

We can verify by converting this back to decimal.

$$ 2^6, 2^5, 2^4, 2^3, 2^2, 2^1, 2^0,  $$
$$ 64, 32, 16, 8, 4, 2, 1,  $$
$$ 1 \cdot 64 + 1 \cdot 32 + 1 \cdot 16 + 1 \cdot 8
 + 0 \cdot 4 + 1 \cdot 2 + 1 \cdot 1 = 123$$

}


\subproblem $0.5_{10}$ \solution{\boxed{0.1_2}

Here, again, we represent a fractional number so we need to use places
that have values less than one.

This case can be done by noticing that 0.5 is exactly one-half which can
be is the value of the binary place immediately to the right of the
decimal.

$$ \boxed{0.1_2} $$
}

\subproblem $2.75_{10}$ \solution{ \boxed{10.11_2}

Same process, what is the largest power of two that is equal or smaller
than 2.75?  $2^1$ is so we put a one in the $2^1$ place and we have 0.75
left over.  What is the largest power of two that is equal or smaller?
$2^0$ or 1 is larger so we put a zero in the ones place.  $2^{-1}$ is
smaller, so we put a 1 in the $2^{-1}$ place and we have 0.25 left over.
$2^{-2}$ or 1/4 is equal or smaller so we put a 1 in the $2^{-2}$ place
and we have nothing left over.  Writing these digits out, we get:

\boxed{10.11_2}
}


\problem{Go metric}
Why is the metric system easier to work with numerically than the
english system?


\solution{
The metric system is based on power of ten exactly like our decimal
number system.  This makes converting between units like kilometers and
meters a simple division or multiplication by a power of ten.
}




\problem{Time Spent}

Please estimate the amount of time you spent on this homework.

\end{document}

%\rule{1cm}{0.5pt}\ \rule{1cm}{0.5pt} .\ \rule{1cm}{0.5pt}\ \rule{1cm}{0.5pt}

\problem{Scientific notation}

Express our current estimate of the world population (7.32 billion) in
scientific notation.

Express the


Significant figures

How many significant figures are reasonable when you express the weight
of a person?

How many significant figures are reasonable when you express a salary?

1. 	Write the following numbers in exponential notation:
0.0021
1,256,000
45,321 (three significant figures)
0.000200
29,001,214
16,200,000,000,000,000,000
6
60
avagadro's number
one google

2. 	Write the following numbers in plain decimal notation:
$3.2 \cdot 10^6$
$5 \cdot 10^2$
$177 \cdot 10^6$
$0.3 x 10^3$
0.004 x 106

3. 	State how many significant figures there are in the following numbers:
32
21,415.40
0.00085
0.004300
4.2

Quantities

If you drive 20 miles, how many kilometers is that?

How many cups of water in a gallon?

My car is traveling at 30 miles per hour, how many meters per second is
that?

You are stranded on a desert island with nothing but dice and want to
create a number system.  How would this work?

Estimate how much time you spent on this homework.



Homework 2

- units
- unit conversion

Homework 3

Homework 4


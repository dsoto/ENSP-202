\documentclass{article}
\usepackage{soto-homework}

% set up whether we are printing assignment or solution
\newif\ifsolution
\input{solution}

\newcommand{\ufrac}[2]{\frac{\textrm{#1}}{\textrm{#2}}}

\begin{document}

\chead{ENSP 202 Homework 2}
\chead{Due Date:  Mon 10 Feb 2014}
\hrule
\vspace{10pt}



For all the following examples, explain your strategy in a few
sentences, and clearly show your work including intermediate steps.
Solutions that only provide the answer without any supporting
explanation or calculations will not receive credit.  It should be
clear from your work that you understand the material.

\problem{Exponential Notation}
Write the following numbers in exponential notation, choosing your
preference of either scientific or engineering notation:

\subproblem 0.0021 \solution{ \boxed{2.1 \cdot 10^{-3}}
}

\subproblem 1,256,000 \solution{ \boxed{ 1.256 \cdot 10^{6}}

}


\subproblem 0.0002 \solution{\boxed{2.0 \cdot 10^{-4}}}

\subproblem 29,000,000 \solution{\boxed{2.9 \cdot 10^{7}}}


\subproblem 16,200,000,000,000,000,000 \solution{\boxed{1.62 \cdot
10^{19}}}


\subproblem Avagadro's Number \solution{\boxed{6.02 \cdot 10^{23}}}

\subproblem One Googol \solution{\boxed{ 1 \cdot 10^{100}}}


\problem{Calculator or Not?}

Show how you would work the following problems without a calculator and
with a calculator.  First, show how you would calculate each of these
using the notation from class ($a \cdot c \cdot 10^{b+d}$).  Second,
perform the calculations on your calculator and write down which keys
you press to create exponents, and what the result is on your
calculator.

\subproblem $2 \cdot 10^3 \times 2 \cdot 10^3$ \solution{\boxed{ 4 \cdot
10^6}}

\subproblem 3 million $\times$ 4 billion \solution{\boxed{12 \cdot 10^{15}}

Write out in exponential notation first
$$ 3 \cdot 10^6 \times 4 \cdot 10^9 $$
$$ 3 \cdot 4 \times 10^{6} \cdot 10^{9} $$
$$ 3 \cdot 4 \cdot 10^{6+9} $$
$$ \boxed{12 \cdot 10^{15}} $$
}


\subproblem $10^9 \div (300 \cdot 10^6)$ \solution{\boxed{3.33}}

\subproblem $ 6.1 \cdot 10^7 \times 3.8 \cdot 10^{-9} $
\solution{\boxed{0.23}}

\subproblem $ 3.5 \cdot 10^{-4} \div (7.2 \cdot 10^{-8}) $
\solution{\boxed{4.86 \cdot 10^3}}

\problem{Notation Preference}

Based on your personal preference would you rather use scientific or
engineering notation?  Why do you prefer your choice?

\solution{

I'm not looking for a correct answer here.  For me, it is a question of
what will make the number the most clear for my audience.  Since I often
work with physical quantities, I'll use engineering notation so that I
can convert to units more easily.  For example the world energy
consumption is about $480 \cdot 10^{18}$ Joules.  Since the metric
prefix for $10^{18}$ is ``exa'' I can also say 480 exajoules.  In this
way, I prefer engineering notation.

}

\problem{Time Spent}

Please estimate the amount of time you spent on this homework.

\end{document}

\problem{Scientific notation}

Express our current estimate of the world population (7.32 billion) in
scientific notation.

Express the


Significant figures

How many significant figures are reasonable when you express the weight
of a person?

How many significant figures are reasonable when you express a salary?

2. 	Write the following numbers in plain decimal notation:
$3.2 \cdot 10^6$
$5 \cdot 10^2$
$177 \cdot 10^6$
$0.3 x 10^3$
0.004 x 106

3. 	State how many significant figures there are in the following numbers:
32
21,415.40
0.00085
0.004300
4.2

Quantities

If you drive 20 miles, how many kilometers is that?

How many cups of water in a gallon?

My car is traveling at 30 miles per hour, how many meters per second is
that?

You are stranded on a desert island with nothing but dice and want to
create a number system.  How would this work?

Estimate how much time you spent on this homework.



Homework 2

- units
- unit conversion

Homework 3

Homework 4

